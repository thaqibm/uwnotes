\documentclass[16pt,a4paper]{article}
\usepackage[utf8]{inputenc}
\usepackage{amsmath}
\usepackage{amsfonts}
\usepackage{amssymb}
\usepackage{amsthm}
\usepackage{physics}


\usepackage[many]{tcolorbox}
\tcbuselibrary{skins,breakable}
\usepackage{hyperref}
\usepackage{mathtools}
\DeclarePairedDelimiter\ceil{\lceil}{\rceil}
\DeclarePairedDelimiter\floor{\lfloor}{\rfloor}

\usepackage{dsfont}

\hypersetup{
    colorlinks=true,
    linkcolor=blue,
    filecolor=magenta,      
    urlcolor=cyan,
    bookmarks=true,
    pdfauthor=Thaqib,
    bookmarksopen=true
}

\newtcbtheorem{defn}{Definition}{
    width=\textwidth,
    colback=white!20,
    colframe=orange,
    colbacktitle=orange,
    fonttitle=\bfseries,
    sharp corners,
    boxrule=1pt,
    breakable,
    enhanced,
    boxed title style={sharp corners},
    attach boxed title to top left
}{def}



\newtcbtheorem{axm}{Axiom}{
    width=\textwidth,
    colback=white!20,
    colframe=black,
    colbacktitle=black,
    fonttitle=\bfseries,
    sharp corners,
    boxrule=1pt,
    breakable,
    enhanced jigsaw,
    boxed title style={sharp corners},
    attach boxed title to top left
}{axm}

\newtcbtheorem{prop}{Proposition}{
    width=\textwidth,
    colback=white!20,
    colframe=black,
    colbacktitle=black,
    fonttitle=\bfseries,
    sharp corners,
    boxrule=1pt,
    breakable,
    enhanced jigsaw,
    boxed title style={sharp corners},
    attach boxed title to top left
}{prop}

\newtcbtheorem{thm}{Theorem}{
    width=\textwidth,
    colback=deepblue!2,
    colframe=deepblue,
    colbacktitle=deepblue,
    fonttitle=\bfseries,
    sharp corners,
    boxrule=1pt,
    breakable,
    enhanced,
    boxed title style={sharp corners},
    attach boxed title to top left
}{thm}

\newtcbtheorem{lemm}{Lemma}{
    width=\textwidth,
    colback=deepred!0,
    colframe=deepred,
    colbacktitle=deepred,
    fonttitle=\bfseries,
    sharp corners,
    boxrule=1pt,
    breakable,
    enhanced,
    boxed title style={sharp corners},
    attach boxed title to top left
}{lemm}



\newtcbtheorem{coll}{Corollary}{
    width=\textwidth,
    colback=white!20,
    colframe=dkgreen,
    colbacktitle=dkgreen,
    fonttitle=\bfseries,
    sharp corners,
    boxrule=1pt,
    breakable,
    enhanced,
    boxed title style={sharp corners},
    attach boxed title to top left
}{thm}




\usepackage{setspace}
\setstretch{1.7}
\usepackage{graphicx}
\usepackage[left=2cm,right=2cm,top=2cm,bottom=2cm]{geometry}

\usepackage{listings}
\usepackage{color}
\definecolor{dkgreen}{rgb}{0,0.3,0}
\definecolor{gray}{rgb}{0.5,0.5,0.5}
\definecolor{mauve}{rgb}{0.58,0,0.82}

\definecolor{deepblue}{rgb}{0,0,0.5}
\definecolor{deepred}{rgb}{0.6,0,0}
\definecolor{deepgreen}{rgb}{0,0.5,0}
\lstset{frame=tb,
  language=python,
  aboveskip=2mm,
  belowskip=2mm,
  showstringspaces=false,
  columns=flexible,
  basicstyle={\linespread{0.9}\small	tfamily},
  numbers=none,
  numberstyle=	iny\color{gray},
  keywordstyle=\color{blue},
  commentstyle=\color{dkgreen},
  stringstyle=\color{deepred},
  breaklines=true,
  breakatwhitespace=true,
  tabsize=4
}

\theoremstyle{definition}
\newtheorem{definition}{Definition}[section]
\newtheorem{example}{Example}

\newtheorem{theorem}{Theorem}[section]
\newtheorem{corollary}{Corollary}[theorem]
\newtheorem{lemma}[theorem]{Lemma}


\newcommand{\ang}[1]{\langle #1 \rangle}

\newcommand{\OR}{\vee}
\newcommand{\Z}{\mathbb{Z}}
\newcommand{\C}{\mathbb{C}}
\newcommand{\R}{\mathbb{R}}


\newcommand{\AND}{\wedge}
\author{Thaqib Mo.}
\title{ Polynomials }
\begin{document}
\maketitle
\newpage
\section{Complex Numbers}
Using localization similar to the construction of $\mathbb{Q}$ we can construct elements of $\C$ in terms of ordered pairs $(a,b)\in \R\times \R$. The addition is defined component wise, 
\[(a,b) + (c,d) = (a+c,b+d)\]
and multiplication similar to the Gaussian integers is defined as
\[(a,b)\cdot (c,d) = (ac-bd, ad+bc)\]
The addition is the same as the addition in $\R$ so we can already conclude that $\C,+$ is an abelian group. $(1,0)$ is clearly the multiplicative identity and it is easy to check that multiplication is associative. 
\\
To check that every non-zero element in $\C$ has an multiplicative inverse: 
\begin{align*}
\frac{1}{a+bi} = \frac{a-bi}{a^2 + b^2} = \frac{a}{a^2+b^2} - \frac{b}{a^2+b^2}i
\end{align*}
So we have $\left( \frac{a}{a^2+b^2}, \frac{b}{a^2+b^2}\right)$ this ordered pair is the inverse of $(a,b)$. We have $a^2 + b^2 \neq 0$ when $(a,b)\neq (0,0)$. Now consider 
\begin{align*}
& (a,b)\cdot ((c,d)+(e,f)) = (a,b)\cdot (c+e,d+f) \\
& = \left(a(c+e) - b(d+f), a(d+f) + b(c+e)\right) \\
& = \left(ac- bd + ae -bf, ad+ bc +af +be\right) \\
& = (a,c)\cdot (e,f) + (a,b)\cdot (e,f)
\end{align*} 
So we can now say $\C$ is a field. 

\subsection{Complex Number Constructions and Properties}
\begin{defn}{}{}
Let $z\in\C$ and we write $z=a+bi$ for some $a,b \in \R$
\begin{itemize}
\item The form $a+bi$ is called the standard form of $z$ 
\item The number $a$ is the \emph{Real} part of $z$ denoted by $\Re(z)$ 
\item The number $b$ is the \emph{imaginary} part of $z$ denoted by $\Im(z)$
\item $\bar{z} = a-bi$ is called the \emph{complex conjugate}
\item $|z| = \sqrt{a^2 + b^2}$ is called the \emph{absolute value} of $z$
\end{itemize}
\end{defn}

\begin{prop}{}{}
$\phi:\C \rightarrow \C$ given by $\phi(z) = \bar{z}$ is a ring homomorphism. 
\end{prop}
\begin{proof}
Let $z=a+bi$ and $w=c+di$ \\
Consider $\phi(z+w) = \overline{z+w} = a+c -(b+d)i  = a-bi + c-di = \phi(z) + \phi(w)$. 
\\
$\phi(zw) = \bar{zw} = \overline{ac-bd + (ad+bc)i} = ac-bd -(ad+bc)i = ac-bd-adi-bci = (a-bi)(c-di) = \phi(z)\phi(w)$. For $\phi(1+0i) = 1-0i = 1$. So $\phi$ is a ring homomorphism. 
\end{proof}


\begin{prop}{}{}
For all $z\in \C$ we have $|zw| = |z||w|$ 
\end{prop}
\begin{proof}
Let $z=a+bi$ and $w=c+di$ we have $zw = ac-bd +(ad+bc)i$ so we have 
\begin{align*}
|zw| &= \sqrt{(ac-bd)^2 (ad+bc)^2} \\
& = \sqrt{a^2c^2 -2acbd + b^2d^2 + a^2d^2 +2adbc + b^2c^2}\\
&=\sqrt{a^2c^2  + b^2d^2 + a^2d^2 + b^2c^2}\\
&= \sqrt{a^2(c^2+d^2) + b^2(c^2+d^2)}\\
&= \sqrt{(a^2+ b^2)(c^2+d^2)}\\
&= \sqrt{a^2+b^2}\sqrt{c^2+d^2}\\
&= |z||w|
\end{align*}

\end{proof}

The \emph{triangle inequality} also holds in $\C$
\begin{thm}{Triangle inequality}{}
For all $z,w\in \C$ we have $|z+w|\leq |z| + |w|$
\end{thm}
\begin{proof}
\begin{align*}
|z+w|^2 &= (z+w)\overline{(z+w)}\\
&= (z+w)(\bar{z}+\bar{w}) \\
&= z\bar{z} + z\bar{w} + w\bar{z} + w\bar{w} \\
& = |z|^2 + |w|^2 + (z\bar{w} + \overline{z\bar{w}}) \\
& = |z|^2 + |w|^2 + 2\Re(z\bar{w})
\end{align*}
Note that $\Re(z\bar{w})\leq |z\bar{w}| = |z||\bar{w}| = |z||w|$. 
So we have \[|z+w|^2 = |z|^2 + |w|^2 + 2\Re(z\bar{w}) \leq |z|^2 + 2|z||w| + |w|^2 = (|z|+|w|)^2 \]
Taking the positive square roots gives the triangle in equality.   
\end{proof}
\subsection{Polar Form of a Complex Number}











































































































\end{document}
