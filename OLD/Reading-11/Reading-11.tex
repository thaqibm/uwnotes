
\documentclass[16pt,a4paper]{article}
\usepackage[utf8]{inputenc}
\usepackage{amsmath}
\usepackage{amsfonts}
\usepackage{amssymb}
\usepackage{amsthm}

\usepackage{setspace}
\setstretch{1.7}
\usepackage{graphicx}
\usepackage[left=2cm,right=2cm,top=2cm,bottom=2cm]{geometry}





\usepackage{hyperref}

\hypersetup{
    colorlinks=true,
    linkcolor=blue,
    filecolor=magenta,      
    urlcolor=cyan,
    pdftitle={Sharelatex Example},
    bookmarks=true,
    pdfpagemode=FullScreen,
}



\usepackage[many]{tcolorbox}
\tcbuselibrary{skins,breakable}


\newtcbtheorem{defn}{Definition}{
    width=\textwidth,
    colback=white!20,
    colframe=orange,
    colbacktitle=orange,
    fonttitle=\bfseries,
    sharp corners,
    boxrule=1pt,
    breakable,
    enhanced,
    boxed title style={sharp corners},
    attach boxed title to top left
}{def}


\newtcbtheorem{axm}{Axiom}{
    width=\textwidth,
    colback=white!20,
    colframe=black,
    colbacktitle=black,
    fonttitle=\bfseries,
    sharp corners,
    boxrule=1pt,
    breakable,
    enhanced jigsaw,
    boxed title style={sharp corners},
    attach boxed title to top left
}{axm}


\newtcbtheorem{thm}{Theorem}{
    width=\textwidth,
    colback=blue!10,
    colframe=blue,
    colbacktitle=blue,
    fonttitle=\bfseries,
    sharp corners,
    boxrule=1pt,
    breakable,
    enhanced,
    boxed title style={sharp corners},
    attach boxed title to top left
}{thm}




\newtcbtheorem{coll}{Corollary}{
    width=\textwidth,
    colback=white!20,
    colframe=red,
    colbacktitle=red,
    fonttitle=\bfseries,
    sharp corners,
    boxrule=1pt,
    breakable,
    enhanced,
    boxed title style={sharp corners},
    attach boxed title to top left
}{coll}











\usepackage{setspace}
\setstretch{1.7}
\usepackage{graphicx}
\usepackage[left=2cm,right=2cm,top=2cm,bottom=2cm]{geometry}

\usepackage{listings}
\usepackage{color}
\definecolor{dkgreen}{rgb}{0,0.6,0}
\definecolor{gray}{rgb}{0.5,0.5,0.5}
\definecolor{mauve}{rgb}{0.58,0,0.82}





\usepackage{listings}
\usepackage{color}
\definecolor{dkgreen}{rgb}{0,0.6,0}
\definecolor{gray}{rgb}{0.5,0.5,0.5}
\definecolor{mauve}{rgb}{0.58,0,0.82}

\definecolor{deepblue}{rgb}{0,0,0.5}
\definecolor{deepred}{rgb}{0.6,0,0}
\definecolor{deepgreen}{rgb}{0,0.5,0}
\lstset{frame=tb,
  language=python,
  aboveskip=2mm,
  belowskip=2mm,
  showstringspaces=false,
  columns=flexible,
  basicstyle={\linespread{0.9}\small	tfamily},
  numbers=none,
  numberstyle=	iny\color{gray},
  keywordstyle=\color{blue},
  commentstyle=\color{dkgreen},
  stringstyle=\color{deepred},
  breaklines=true,
  breakatwhitespace=true,
  tabsize=4
}

\theoremstyle{definition}
\newtheorem{definition}{Definition}[section]

\newtheorem{theorem}{Theorem}[section]
\newtheorem{corollary}{Corollary}[theorem]
\newtheorem{lemma}[theorem]{Lemma}


\newcommand{\OR}{\vee}

\newcommand{\AND}{\wedge}

\author{Thaqib Mo.}
\title{ Equivalence Relations, Equivalence Classes, and Partitions }
\begin{document}
\maketitle
\newpage
\section{Equivalence Relations}

\begin{defn}{Equivalence relation}{}
Let $R$ be be a binary relation on set $A$.  We say that 
\begin{itemize}
\item $R$ is \textit{reflexive} if , for all $a\in R$, we have $aRa$
\item $R$ is \textit{symmetric} if , for all $a,b \in A$  if $aRb$ then $bRa$
\item $R$ is \textit{transitive} if , for all $a,b,c \in A$ if $aRb$ and $bRc$ then $aRc$. 
\item $R$ is an equivalence relation if $R$ is reflexive, symmetric and transitive. 
\end{itemize}
\end{defn}

\subsection{Equivalence Classes}
\begin{defn}{Equivalence Class}{}
Suppose that $E$ is an equivalence relation on some set $A$. Given an element $a\in A$ the equivalence class of $a$ modulo $E$ is the set
\[[a]_E = \{x\in A : aEx\}\]
\end{defn}
So a equivalence relation splits up a set $A$ into smaller equivalence classes.  For any 2 elements in $A$ their equivalence classes are identical or they are disjoint. 
\begin{thm}{}{}
Let $E$ be an equivalence relation on set $A$. 
\begin{itemize}
\item[(1)] We have $aEb$ if and only if $[a] = [b]$
\item[(2)] We have $(a,b) \notin E$ if and only if $[a]\cap [b] = \varnothing$
\end{itemize}
\end{thm}

\begin{proof}
Assume that $aEb$. We have to prove that $[a] = [b]$. \\
Let $x\in [a]$ by definition we have $aEx$, since $E$ is symmetric and we know $aEb$ then $bEa$ as well. As $E$ is transitive and we have $bEa$ and $aEx$ this implies $bEx$. So we get $x\in [b]$ showing $[b] \subset [a]$ a symmetric argument for $[a] \subset [b]$ follows. Proving that $[a] = [b]$ is $aEb$. \\
 Next, assume $[a] = [b]$. Since $E$ is reflexive we know that $bEb$ so we have $b\in [b]$ but we have $[b] = [a]$ then we get $b\in [a]$ by definition we get $aEb$. 
 
 
We can prove $(2)$ if we have $[a] \cap [b] = \varnothing$ then $[a] \neq [b]$ because both $[a]$ and $[b]$ are non empty sets. On the other hand assume $[a] \cap [b] \neq \varnothing$. Then there is some $x \in [a] \cap [b]$ so we have $x \in [a]$ and $x\in [b]$. That is by definition $aEx$ and $bEx$ using the property of equivalence relations we can get to $aEb$. By part (1) we have $[a] = [b]$. 
\end{proof}




\newpage
\section{Partitions}
\begin{defn}{Partitions}{}
Given any set $A$, a \textit{partition} $\mathcal{P}$ of $A$ is a collection of  non empty sets with the properties:
\begin{itemize}
\item[(1)] For any two distinct sets $P_1, P_2 \in \mathcal{P}$, we have $P_1 \cap P_2 = \varnothing$
\item[(2)] $\bigcup \mathcal{P} = A$
\end{itemize}{} 
\end{defn}
Lemma $1.1$ essentially shows that a every equivalence relation on $A$ gives us  a set $A$ gives us a partition of that set. We have the notation $A/E$ for the set $\{[a]_E : a \in A\}$. 
\begin{thm}{}{}
Let $E$ be a equivalence relation on $A$. Then $A/E$ is a partition of $A$. 
\end{thm}
\begin{proof}
Every equivalence class $[a]$ is non empty as we must have $a \in [a]$. According to \textbf{lemma 1.1} if 2 equivalence classes are not equal they are disjoint. Which verifies condition (1) for partitions. Condition (2) follows from the fact that every $a\in A$ belongs to its equivalence class $[a]$, so the union of all equivalence classes $[a]$ is the set whole set $A$
\end{proof}
Therefore \emph{every equivalence relation gives rise to a partition.} We can also show that the converse is true: That every partition can be used to define an equivalence relation on that set. 

\begin{thm}{}{}
Let $A$ be a set , and let $\mathcal{P}$ be a partition on that set. We define a relation $E$ on $A$ by stating $a_1 E a_2$ if there is some $P \in \mathcal{P}$ such that $a_1 \in P$ and $a_2 \in P$. Then $E$ is an equivalence relation on $A$. 
\end{thm}
\begin{proof}
$\;$\\
\begin{itemize}
\item \textbf{Reflexivity} Let $a\in A$ be arbitrary. Since $\mathcal{P}$ is a partition, there is some $P \in \mathcal{P}$ containing $a$. So clearly $a\in P$ and $a\in P$ then we have $aEa$ for every $a\in A$. 

\item \textbf{Symmetry} Suppose we have $a_1, a_2 \in A$ such that $a_1 E a_2$. There there is some $P \in \mathcal{P}$ such that $a_1 \in P$ and $a_2 \in P$. Symmetrically we also have $a_2 \in P$ and $a_1 \in P$ leading to $a_2 E a_1$

\item \textbf{Transitivity} Suppose we have $a_1, a_2, a_3 \in A$ such that $a_1Ea_2$ and $a_2Ea_3$. Then for some $P_1\in\mathcal{P}$ we have $a_1 \in P_1$ and $a_2 \in P_1$, and also for some $P_2 \in \mathcal{P}$ we have $a_2 \in P_2$ and $a_3 \in P_2$. Since $\mathcal{P}$ is a partition if $P_1$ and $P_2$ were distinct we would have $P_1 \cap P_2 = \varnothing$. But this is not the case as $a_2$ is in both sets. Therefore we have $P_1 = P_2$. So $a_1$ and $a_3$ are in the same $P_1 \in \mathcal{P}$ leading to $a_1 E a_3$ 

\end{itemize}
\end{proof} 
\newpage
Therefore, we see that this correspondence runs both ways: every equivalence relation gives a partition, and every partition gives an equivalence relation.


\begin{defn}{}{}
Let $A$ be a set and $E$ be an equivalence relation on $A$.  A set $X$ is called the \textit{set of representatives} for $E$ if $X$ contains exactly one element from each equivalence class. \\
In other words, for each $[a] \in A/E$ we have $X\cup [a] = \{\alpha\}$ for some $\alpha \in [a]$. 
\end{defn}























































































































































\end{document}
