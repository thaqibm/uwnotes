
\documentclass[16pt,a4paper]{article}
\usepackage[utf8]{inputenc}
\usepackage{amsmath}
\usepackage{amsfonts}
\usepackage{amssymb}
\usepackage{amsthm}

\usepackage{hyperref}

\hypersetup{
    colorlinks=true,
    linkcolor=blue,
    filecolor=magenta,      
    urlcolor=cyan,
    pdftitle={Sharelatex Example},
    bookmarks=true,
    pdfpagemode=FullScreen,
}



\usepackage[many]{tcolorbox}
\tcbuselibrary{skins,breakable}


\newtcbtheorem{defn}{Definition}{
    width=\textwidth,
    colback=white!20,
    colframe=orange,
    colbacktitle=orange,
    fonttitle=\bfseries,
    sharp corners,
    boxrule=1pt,
    breakable,
    enhanced,
    boxed title style={sharp corners},
    attach boxed title to top left
}{def}


\newtcbtheorem{axm}{Axiom}{
    width=\textwidth,
    colback=white!20,
    colframe=black,
    colbacktitle=black,
    fonttitle=\bfseries,
    sharp corners,
    boxrule=1pt,
    breakable,
    enhanced jigsaw,
    boxed title style={sharp corners},
    attach boxed title to top left
}{axm}


\newtcbtheorem{thm}{Theorem}{
    width=\textwidth,
    colback=blue!10,
    colframe=blue,
    colbacktitle=blue,
    fonttitle=\bfseries,
    sharp corners,
    boxrule=1pt,
    breakable,
    enhanced,
    boxed title style={sharp corners},
    attach boxed title to top left
}{thm}




\newtcbtheorem{coll}{Corollary}{
    width=\textwidth,
    colback=white!20,
    colframe=red,
    colbacktitle=red,
    fonttitle=\bfseries,
    sharp corners,
    boxrule=1pt,
    breakable,
    enhanced,
    boxed title style={sharp corners},
    attach boxed title to top left
}{coll}











\usepackage{setspace}
\setstretch{1.7}
\usepackage{graphicx}
\usepackage[left=2cm,right=2cm,top=2cm,bottom=2cm]{geometry}

\usepackage{listings}
\usepackage{color}
\definecolor{dkgreen}{rgb}{0,0.6,0}
\definecolor{gray}{rgb}{0.5,0.5,0.5}
\definecolor{mauve}{rgb}{0.58,0,0.82}

\usepackage{setspace}
\setstretch{1.7}
\usepackage{graphicx}
\usepackage[left=2cm,right=2cm,top=2cm,bottom=2cm]{geometry}

\usepackage{listings}
\usepackage{color}
\definecolor{dkgreen}{rgb}{0,0.6,0}
\definecolor{gray}{rgb}{0.5,0.5,0.5}
\definecolor{mauve}{rgb}{0.58,0,0.82}

\definecolor{deepblue}{rgb}{0,0,0.5}
\definecolor{deepred}{rgb}{0.6,0,0}
\definecolor{deepgreen}{rgb}{0,0.5,0}
\lstset{frame=tb,
  language=python,
  aboveskip=2mm,
  belowskip=2mm,
  showstringspaces=false,
  columns=flexible,
  basicstyle={\linespread{0.9}\small	tfamily},
  numbers=none,
  numberstyle=	iny\color{gray},
  keywordstyle=\color{blue},
  commentstyle=\color{dkgreen},
  stringstyle=\color{deepred},
  breaklines=true,
  breakatwhitespace=true,
  tabsize=4
}

\theoremstyle{definition}
\newtheorem{definition}{Definition}[section]

\newtheorem{theorem}{Theorem}[section]
\newtheorem{corollary}{Corollary}[theorem]
\newtheorem{lemma}[theorem]{Lemma}


\newcommand{\OR}{\vee}

\newcommand{\AND}{\wedge}
\newcommand{\ord}{\preceq}


\author{Thaqib Mo.}
\title{ Reading-12 }
\begin{document}
\maketitle
\newpage
\section{Order Relations}
\begin{defn}{Antisymmetric Relation}{}
Let $R$ be a binary relation on set $A$. We say that $R$ if, whenever we have $a,b \in A$ such that $aRb$ and $bRa$, it follows that $a=b$. A relation $\ord$ is an order relation on $A$ if if it is reflexive , \textit{antisymmetric}, and transitive.
\end{defn}
\subsection{Chains and Extremal Elements}
\begin{defn}{Total ordering}{}
If $\ord$ is a partial ordering on set $A$ we say that 2 elements $a,b \in A$ are comparable if either $a\ord b$ or $b \ord a$. A partial order in which every pair of elements is comparable is called a \textit{total ordering} or \textit{linear ordering}. \\
\end{defn}
{\color{red} Example} The relation $\leq$ on $\mathbb{Z}$ is a total ordering as every pair $a,b \in \mathbb{Z}$ are comparable. 

Another important notion which comes is the notion of a \textit{chain}:
\begin{defn}{Chain}{}
If $\ord$ is a partial order on set $A$, a subset $C$ of $A$ is called a \textit{chain} if every pair of $C$ are comparable. In particular, if $\ord$ is a total order on $A$ then $A$ itself is a chain in $A$. 
\end{defn}

Because not every element of a set must be comparable we have to be careful in distinguishing the greatest and the least elements of of a given set. The distinction is given by the following definitions. 
\begin{defn}{Maximal, minimal, greatest, least elements}{}
Let $A$ be a set with partial order relation $\ord$. Given a subset $B$ of $A$ we say that 
\begin{itemize}
\item An element $b\in B$ is the \textbf{least element} of $B$ if we have $b\ord b^\prime$ for all $b^\prime \in B$. Similarly, an element is the \textbf{greatest element} if $b^\prime \ord b$ for all $b^\prime\in B$. 

\item An element $b\in B$ is the \textbf{minimal element} of $B$ if there are no smaller elements, that is if $b^\prime \ord b$ \textbf{for some} $b^\prime \in B$ then $b = b^\prime$. Similarly an element is the \textbf{maximal element} if there are no larger elements, that is if $b \ord b^\prime$ \textbf{for some} $b^\prime \in B$ then $b^\prime = b$. 
\end{itemize}
\end{defn}

\newpage
\section{Bounds on Sets, Suprema, and Infima}
\begin{defn}{Bounds, Suprema, and Infima}{}
Suppose $A$ is a set with order relation $\ord$ and with a subset $B$. 
\begin{itemize}
\item An element $a\in A$ is a lower bound for $B$ if $a\ord b$ \textbf{for all} $b\in B$. Similarly $a\in A$ is a upper bound for $B$ if $b \ord a$ \textbf{for all} $b\in B$. 

\item An element $a\in A$ is the infimum (Greatest lower bound) of $B$ if $a$ is the greatest element of the set of al lower bounds for $B$. 

\item Am element $a\in A$ is the supremum (Least upper bound) of $B$ if $a$ is the least element of the set of all upper bounds for $B$. 
\\
If they exist, we use $sup\;B$ and $inf \;B$ to denote the supremum and infimum of a set $B$, respectively.
 
\end{itemize}
\end{defn}


























































\end{document}
