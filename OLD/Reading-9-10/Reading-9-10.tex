
\documentclass[16pt,a4paper]{article}
\usepackage[utf8]{inputenc}
\usepackage{amsmath}
\usepackage{amsfonts}
\usepackage{amssymb}
\usepackage{amsthm}



\usepackage{hyperref}

\hypersetup{
    colorlinks=true,
    linkcolor=blue,
    filecolor=magenta,      
    urlcolor=cyan,
    pdftitle={Sharelatex Example},
    bookmarks=true,
    pdfpagemode=FullScreen,
}



\usepackage[many]{tcolorbox}
\tcbuselibrary{skins,breakable}


\newtcbtheorem{defn}{Definition}{
    width=\textwidth,
    colback=white!20,
    colframe=orange,
    colbacktitle=orange,
    fonttitle=\bfseries,
    sharp corners,
    boxrule=1pt,
    breakable,
    enhanced,
    boxed title style={sharp corners},
    attach boxed title to top left
}{def}


\newtcbtheorem{axm}{Axiom}{
    width=\textwidth,
    colback=white!20,
    colframe=black,
    colbacktitle=black,
    fonttitle=\bfseries,
    sharp corners,
    boxrule=1pt,
    breakable,
    enhanced jigsaw,
    boxed title style={sharp corners},
    attach boxed title to top left
}{axm}


\newtcbtheorem{thm}{Theorem}{
    width=\textwidth,
    colback=blue!10,
    colframe=blue,
    colbacktitle=blue,
    fonttitle=\bfseries,
    sharp corners,
    boxrule=1pt,
    breakable,
    enhanced,
    boxed title style={sharp corners},
    attach boxed title to top left
}{thm}




\newtcbtheorem{coll}{Corollary}{
    width=\textwidth,
    colback=white!20,
    colframe=red,
    colbacktitle=red,
    fonttitle=\bfseries,
    sharp corners,
    boxrule=1pt,
    breakable,
    enhanced,
    boxed title style={sharp corners},
    attach boxed title to top left
}{coll}











\usepackage{setspace}
\setstretch{1.7}
\usepackage{graphicx}
\usepackage[left=2cm,right=2cm,top=2cm,bottom=2cm]{geometry}

\usepackage{listings}
\usepackage{color}
\definecolor{dkgreen}{rgb}{0,0.6,0}
\definecolor{gray}{rgb}{0.5,0.5,0.5}
\definecolor{mauve}{rgb}{0.58,0,0.82}

\definecolor{deepblue}{rgb}{0,0,0.5}
\definecolor{deepred}{rgb}{0.6,0,0}
\definecolor{deepgreen}{rgb}{0,0.5,0}
\lstset{frame=tb,
  language=python,
  aboveskip=2mm,
  belowskip=2mm,
  showstringspaces=false,
  columns=flexible,
  basicstyle={\linespread{0.9}\small	tfamily},
  numbers=none,
  numberstyle=	iny\color{gray},
  keywordstyle=\color{blue},
  commentstyle=\color{dkgreen},
  stringstyle=\color{deepred},
  breaklines=true,
  breakatwhitespace=true,
  tabsize=4
}

\theoremstyle{definition}
\newtheorem{definition}{Definition}

\newtheorem{theorem}{Theorem}[section]
\newtheorem{prop}{Proposition}[definition]
\newtheorem{lemma}[theorem]{Lemma}


\newcommand{\OR}{\vee}

\newcommand{\AND}{\wedge}

\author{Thaqib Mo.}
\title{ Reading-9-10 }
\begin{document}
\maketitle
\newpage
\section{Ordered Pairs, Relations, and Cartesian Products}
\subsection{Ordered Pairs}
An ordered paid $(x,y)$ as the name suggest is ordered unlike sets. So an ordered paid $(x,y) \neq (y,x)$ the ordering of the elements matter. We can only have 2 ordered pairs equal $(x,y) = (x^\prime, y^\prime) \iff x = x^\prime\quad y = y^\prime$. \\
\begin{defn}{Ordered pair}{}
Given any 2 sets $x$ and $y$, the \textit{ordered paid} $(x,y)$ is defined to be the set \[\{\{x\}, \{x,y\}\}\]
\end{defn}
\begin{thm}{Uniqueness of ordered pairs}{}
For any sets $x,y,x_1, y_1$ we have $(x,y) = (x_1, y_1)$ if and only if $x = x_1$ and $y = y_1$. 
\end{thm}
\begin{proof}
($\Leftarrow$)\\
Assume $x = x_1$ and $y = y_1$
Then we have:
\begin{align*}
(x,y) = \{\{x\}, \{x,y\}\}\\
= \{\{x_1\}, \{x_1,y_1\}\}\\
= (x_1, y_1)
\end{align*}
($\Rightarrow$)
Now assume that $(x,y) = (x_1, y_1)$ due to the assumption we have
\[\{\{x\}, \{x,y\}\} = \{\{x_1\}, \{x_1,y_1\}\}\]
By axiom of Extensionality, 2 sets are equal if and only if they have the same elements. 
\begin{itemize}
\item Case 1: $x = y$
Then we have $\{\{x\}, \{x,y\}\} =  \{\{x\}\}$ by the above equality we must have $\{x\} = \{x_1\} \iff x = x_1$, and $\{x_1, y_1\} = \{x\}$ this is true if and only if $x = x_1$ and $x = y_1  = y$. There we have established $x = x_1$ and $y = y_1$
\item Case 2: $x \neq y$
Given the equality above $\{x,y\} = \{x_1\}$ is impossible then we must have $\{x,y\} = \{x_1, y_1\}$. Then it must be the case that $\{x\} = \{x_1\} \iff x = x_1$. 
\[\{x,y\} = \{x_1, y_1\} = \{x,y_1\}\]
Then we must have $y_1 = y$ this completes the proof. 
\end{itemize}
\end{proof}


\newpage
\subsection{Cartesian Products}
Informally the product of 2 sets $X\times Y$ is the collection of all ordered pairs. With first coordinate from $X$ and second from $Y$. In order to construct this product we need to supply a set with all elements of $X\times Y$. This can be defined as
\[X\times Y = \{w\in Z : w = (x,y) \text{ for some } x\in X, y\in Y\}\]   
The problem is what is this set $Z$?\\
First we need a set that contains the elements of $X$ and $Y$ this can be done with axiom of union. By constructing $X\cup Y$. Certainly $\{x\}$ and $\{x,y\}$ are both subsets of $X\cup Y$. So the elements of the set $\{x, \{x,y\}\}$ can be taken to belong to the power set $\mathcal{P(}X\cup Y)$\\
Now we have $\{x\} \in $ $\mathcal{P(}X\cup Y)$ and $\{x,y\} \in\mathcal{P(}X\cup Y)$. but the set $\{x, \{x,y\}\}$ is not a subset of $X\times Y$ so we cannot take $ Z = \mathcal{P(}X\cup Y)$. \\
To get a set of \textit{subsets} of $\mathcal{P(}X\cup Y)$ we need to consider the power set of this set. We need an element of $\mathcal{P(}\mathcal{P(}X\cup Y))$. As $\{\{x\}, \{x,y\}\} \in \mathcal{P(}\mathcal{P(}X\cup Y))$. So we can take $Z = \mathcal{P(}\mathcal{P(}X\cup Y))$ The formal definition is :
\begin{defn}{Product of sets}{}
Given any two sets $X$ and $Y$ , the Cartesian product of $X$ and $Y$ is the set
\[X\times Y = \{w\in \mathcal{P(}\mathcal{P(}X\cup Y)) \;:\; w = (x,y) \text{ for some } x\in X, y\in Y\}\]
\end{defn}

\subsection{Relations and Functions}
\begin{defn}{Binary Relation}{}
Given 2 sets a $A$, $B$ \textit{binary} relation from $A$ to $B$ is a subset of $A\times B$. More generally a set $R$ is called a \textit{relation} if all the elements of $R$ are ordered pairs. Rather than writing $(x,y) \in R$ we write $xRy$.
\end{defn}
\begin{defn}{Function}{}
A \textit{function} is a relation such if $(a,b_1)$ and $(a, b_2)$ in $f$ then we must have $b_1 = b_2$. If $(a,b)\in f$ we write $f(a) = b$ 
\end{defn}
A function from the set $A$ to $B$ is denoted by 
\[f:A \rightarrow B\]
\newpage
\subsection{Terminology Around Relations and Functions}
\begin{defn}{Domain and Range}{}
The \textbf{domain} of the relation $R$ is the set of all such $x$ such that $(x,y)\in R$ for some $y$. \\
The \textbf{range} of $R$ is the set of all such $y$ such that $(x,y)\in R$ for some $x$. 
We use ran$(R)$ and dom$(R)$ to denote the domain and range of $R$. 
The set ran$(R)\cup$dom$(R)$ is called the \textbf{\textit{\emph{field}}} of $R$.  
\end{defn}
All these definitions also apply to functions. 
\begin{defn}{Image, Inverse Image}{}
Let $R$ be a binary relation. The \textit{image} of set $A$ under $R$ is the set
\[R(A) = \{b\in \text{ran}(R) : (a,b) \in R \text{ for some } a\in A\}\]
Similarly given set $B$ the \textit{inverse image} of $B$ under $R$ is defined as
\[R^{-1} (B) = \{a\in \text{dom$R$} : (a,b)\in R \text{ for some $a\in A$}\}\]
\end{defn}
Given 2 relations their composition $R_2\circ R_1$ is defined as:
\[R_2 \circ R_1 = \{z\in \text{ dom$(R_2)\times$ ran$(R_1)$}\} : z = (a,c) \text{ such that $\exists b (a,b) \in R_1$ and $(b,c)\in R_2$}\}\]

\section{Invertibility, Injectivity, and Surjectivity}
A function is called Injective if for all $x_1, x_2 \in $dom$f$ such that $x_1 \neq x_2$. We have $f(x_1) \neq f(x_2)$. \\

If a function from $A$ to $B$ is called Surjective (onto) if ran$(f) = B$. In other words, for each $b\in B$ there exits $a\in A$ such that $f(a) = b$. \\

A function is called a bijection if it is both injective and surjective. 































































































































































































\end{document}
