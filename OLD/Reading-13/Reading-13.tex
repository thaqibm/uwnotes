
\documentclass[16pt,a4paper]{article}
\usepackage[utf8]{inputenc}
\usepackage{amsmath}
\usepackage{amsfonts}
\usepackage{amssymb}
\usepackage{amsthm}

\usepackage[many]{tcolorbox}
\tcbuselibrary{skins,breakable}


\newtcbtheorem{defn}{Definition}{
    width=\textwidth,
    colback=white!20,
    colframe=orange,
    colbacktitle=orange,
    fonttitle=\bfseries,
    sharp corners,
    boxrule=1pt,
    breakable,
    enhanced,
    boxed title style={sharp corners},
    attach boxed title to top left
}{def}


\newtcbtheorem{axm}{Axiom}{
    width=\textwidth,
    colback=white!20,
    colframe=black,
    colbacktitle=black,
    fonttitle=\bfseries,
    sharp corners,
    boxrule=1pt,
    breakable,
    enhanced jigsaw,
    boxed title style={sharp corners},
    attach boxed title to top left
}{axm}


\newtcbtheorem{thm}{Theorem}{
    width=\textwidth,
    colback=blue!10,
    colframe=blue,
    colbacktitle=blue,
    fonttitle=\bfseries,
    sharp corners,
    boxrule=1pt,
    breakable,
    enhanced,
    boxed title style={sharp corners},
    attach boxed title to top left
}{thm}




\newtcbtheorem{coll}{Corollary}{
    width=\textwidth,
    colback=white!20,
    colframe=red,
    colbacktitle=red,
    fonttitle=\bfseries,
    sharp corners,
    boxrule=1pt,
    breakable,
    enhanced,
    boxed title style={sharp corners},
    attach boxed title to top left
}{coll}




\usepackage{setspace}
\setstretch{1.7}
\usepackage{graphicx}
\usepackage[left=2cm,right=2cm,top=2cm,bottom=2cm]{geometry}

\usepackage{listings}
\usepackage{color}
\definecolor{dkgreen}{rgb}{0,0.6,0}
\definecolor{gray}{rgb}{0.5,0.5,0.5}

\definecolor{deepblue}{rgb}{0,0,0.5}
\definecolor{deepred}{rgb}{0.6,0,0}
\definecolor{deepgreen}{rgb}{0,0.5,0}
\definecolor{lorange}{HTML}{FF9F40}

\lstset{frame=tb,
  language=python,
  aboveskip=2mm,
  belowskip=2mm,
  showstringspaces=false,
  columns=flexible,
  basicstyle={\linespread{0.9}\small	tfamily},
  numbers=none,
  numberstyle=	iny\color{gray},
  keywordstyle=\color{blue},
  commentstyle=\color{dkgreen},
  stringstyle=\color{deepred},
  breaklines=true,
  breakatwhitespace=true,
  tabsize=4
}

\theoremstyle{definition}
\newtheorem{definition}{Definition}

\newtheorem{Axiom}{Axiom}

\newtheorem{theorem}{Theorem}
\newtheorem{corollary}{Corollary}[theorem]
\newtheorem{lemma}[theorem]{Lemma}


\newcommand{\OR}{\vee}

\newcommand{\AND}{\wedge}

\author{Thaqib Mo.}
\title{ Reading-13 }
\begin{document}
\maketitle
\newpage
\section{Axiom of choice}

Suppose $\mathcal{C}$ is a non empty collection of sets then the Cartesian product of all sets in $\mathcal{C}$ is written as:

\[\prod\limits_{C\in \mathcal{C}} C\] 

If $\mathcal{C}$ is an infinite collection of sets then we need infinite ordered tuple of elements. Then we need a function $\alpha : \mathcal{C} \rightarrow \bigcup \mathcal{C}$. Then for each $C\in \mathcal{C}$ we have $\alpha(C) \in C$. The function $\alpha (C)$ should give the $C$-th coordinate of the infinite tuple. We can formally define this as:\\

\begin{defn}{Cartesian Product for infinite sets}{}
Let $\mathcal{C}$ denote a non empty collection of sets. The Cartesian product $\prod\limits_{C\in \mathcal{C}}C$  is the set of all functions $\alpha : \mathcal{C} \rightarrow \bigcup \mathcal{C}$  with the property $\alpha(C) \in C$.
\end{defn} 

We can use this definition to redefine the Cartesian product, even for finitely many sets. We can define an ordered pair $(a,b) \in A\times B$ as $(\alpha(A), \alpha(B))\in A\times B$. \\
The problem arises when for an infinite number of sets in $\mathcal{C}$ we cannot show (using the standard axioms of set theory) that:

\[\prod_{C\in \mathcal{C}}C \neq \varnothing\]
For this we need a new axiom. 

\begin{axm}{Axiom of Choice}{}
The Cartesian product of any non-empty collection of non-empty sets is
non-empty.
\end{axm}
This axiom is also equivalent to the existence of a choice function for every non-empty collection of set $\mathcal{C}$.   




\newpage

\section{Zorn's Lemma and the Well-Ordering Theorem}

Another statement about partially ordered set is logically equivalent to Axiom of choice. 


\begin{thm}{Zorn's Lemma}{}
Let $A$ be a partially ordered set with order relation $\preceq$. Suppose that every chain in $A$ has an upper bound in $A$, then $A$ has a maximal element. 
\end{thm}


A Corollary that follows from Zorn's Lemma is:

\begin{coll}{}{}
Suppose that $A$ is a partially ordered set with order relation $\preceq$ then every chain $\mathcal{C}$ in $A$ is contained in a maximal chain $\mathcal{M}$. 
\end{coll}
\begin{proof}
Let $\mathcal{C}$ be an arbitrary chain in $A$. Consider the set $\Gamma$ of all chains $\mathcal{C}$ in $A$. The subset relation $\subseteq$ relation on $\mathcal{P}(A)$ giving the order relation between elements of $\Gamma$. Now suppose we have a chain $\mathcal{D}$ in the ordered set $\Gamma$. We can show that $\mathcal{D}$ is an upper bound in $\gamma$
\\

If we define $\mathcal{C}_0 = \bigcup \mathcal{D}$, the union of all the, clearly for each $\mathcal{C}^\prime \in \mathcal{D}$ we have $C^\prime \subseteq \mathcal{C}_0$. Since $C_0$ contains every chain of $\mathcal{D}$, each of which contains $\mathcal{C}$ we know that $\mathcal{C}_0$ contains $\mathcal{C}$. Therefore $C_0$ is an upper bound on $\mathcal{D}$ by definition. If we can verify $C_0 \in \Gamma $ that means $C_0$ is a chain in $A$
\\

Suppose we have  $a_1, a_2 \in \mathcal{C}_0$. By construction each element of $a_1$ and $a_2$ belong to a chain in $\mathcal{D}$, so we can write $a_1 \in \mathcal{C}_1$ and $a_2 \in \mathcal{C}_2$. Since $\mathcal{D}$ is a chain with respect to the relation $\subseteq$, we can either have $\mathcal{C}_1\subseteq \mathcal{C}_2$ or we can have $\mathcal{C}_2\subseteq \mathcal{C}_1$. Without loss of generality we say $\mathcal{C}_1\subseteq \mathcal{C}_2$ . Thus both $a_1, a_2 \in C_2$. Since $\mathcal{C}_2$ is a chain, the elements in $\mathcal{C}_2$ are comparable with $\preceq$, this verifies that $\mathcal{C}_0$ is a chain in $A$, as $C_0 \in \Gamma$. 
\\

Now if we apply Zorns lemma to $\Gamma$ which is ordered by $\subseteq$, there is a chain $\mathcal{M} \in \Gamma$, maximal with respect to the containment relation $\subseteq$. This is a chain in $A$ with respect to the ordering $\preceq$, as it is maximal with with respect to $\subseteq$ all chains $\mathcal{C}$ of $A$ are contained in $\mathcal{M}$ and it is not properly contained in any chain.  


\end{proof}




\newpage
The well ordering theorem is another statement known to be equivalent to axiom of choice. We can define the WOP of $\mathbb{N}$ more generally. 
\begin{defn}{Well ordering}{}
Suppose $A$ is a set with order relation $\preceq$. The ordered set $A$ is said to be well ordered if every non empty subset of $A$ has a least element with respect to the relation $\preceq$
\end{defn}

This leads to the well ordering theorem:

\begin{thm}{Well-Ordering Theorem}{}
Every non empty set has a well ordering. 
\end{thm}

In other words, if $A$ is a non empty set then there is a an order relation $\preceq$ on $A$, such that $A$ is well ordered with respect to $\preceq$. This means that the set $\mathbb{R}$ has as well ordering, not with respect to the relation $\leq$ as this fails for any open interval in $\mathbb{R}$ but according to \textit{well ordering theorem} there exits a relation $\preceq$ on $\mathbb{R}$ on which every subset of $\mathbb{R}$ will have a least element with respect to $\preceq$ which is not the relation $\leq$. 
\\

The 3 statements Axiom of choice, Well-Ordering Theorem, Zorn's Lemma are all logically equivalent.    






















































































\end{document}
